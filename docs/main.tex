\documentclass{article}
\usepackage[margin=1in]{geometry}
\usepackage{parskip}

\usepackage{listings}
\lstset{
  basicstyle=\ttfamily\small,
  breaklines=true,
  frame=single
}
\usepackage{physics}
\usepackage{amsmath}
\usepackage{amssymb}
\usepackage{hyperref}
\hypersetup{
  colorlinks=true,
  linkcolor=blue,
  filecolor=magenta,
  urlcolor=cyan,
  pdftitle={Overleaf Example},
  pdfpagemode=FullScreen,
}


\begin{document}

ANPASS is a program for performing least-squares analysis. It was originally
written in Fortran but has since been translated to Go and then to Rust by Brent
R. Westbrook. This documentation corresponds to the Rust version only. In
principle, ANPASS can be used to solve any polynomial or even linear regression
problem, but in practice it is only used for determining the force constants in
quartic force fields. As such, this will be the running example of usage in this
document.

\section{Overview}
\label{sec:over}

A quartic force field (QFF) is a fourth-order Taylor series expansion of the
internuclear potential energy portion of the Watson Hamiltonian of the form

\begin{equation}
  V = \frac{1}{2}\sum_{ij} F_{ij}\Delta_i\Delta_j +
  \frac{1}{6}\sum_{ijk} F_{ijk}\Delta_i\Delta_j \Delta_k+
  \frac{1}{24}\sum_{ijkl} F_{ijkl}\Delta_i\Delta_j\Delta_k\Delta_l
\end{equation}

This can be rephrased as the matrix equation $XF=V$, where $F$ is the vector of
force constants that minimizes the least squares difference $|XF - V|$, $V$ is a
vector of potential energies, and $X$ is itself a matrix composed of the
elements $X_{ik}$ given by the expression

\begin{equation}
  X_{ik} = \prod_j^N x_{ij}^{e_{jk}}
\end{equation}

In turn, the $x_{ij}$ for the QFF example are the $j$th components of the $i$th
displacement, and the $e_{jk}$ are the $k$th exponent in the $j$th row of
exponents. This is similar to the basic polynomial regression problem given by

\begin{equation}
  A =
  \begin{bmatrix}
    x_1^m & \dots & x_1^2 & x_1 & 1 \\
    x_2^m & \dots & x_2^2 & x_2 & 1 \\
    \vdots & \ddots & \vdots & \vdots & \vdots \\
    x_n^m & \dots & x_n^2 & x_n & 1 \\
  \end{bmatrix}
\end{equation}

\noindent
except that $N = 1$ in the basic form, and $k$ is
$\begin{bmatrix}m & m-1 & \dots & 0\end{bmatrix}$ instead of varying in
arbitrary ways. In other words, the exponents are given by the vector
$\begin{bmatrix}m & m-1 & \dots & 0\end{bmatrix}$ instead of the matrix with
elements $e_{jk}$, and the $x_{ij}$ are really the vector of elements $x_i$.

With this formulation, ANPASS can be used to solve the $m$th-order regression
problem. For example, a linear regression problem for $i$ data points is
represented by $e = \begin{bmatrix}1 & 0\end{bmatrix}$. The one caveat to this
is that the resulting function is only printed in the file \verb|fort.9903| in
the units expected for a QFF, so if you want the raw values back, divide your
coefficient by 4.359813653. A full example of such a fitting is given in
Appendix A.

\section{Input format}

The layout of the ANPASS input file is as follows:

\subsection{Header}

\begin{lstlisting}
!INPUT
TITLE
 H2O 2A1 F12-TZ
PRINT
   99
INDEPENDENT VARIABLES
   3
DATA POINTS
  69   -2
(3F12.8,f20.12)
\end{lstlisting}

\noindent
The only line read by the Rust version is the last. This line is parsed by the
regular expression:

\begin{lstlisting}
"(?i)^\s*\((\d+)f[0-9.]+,f[0-9.]+\)\s*$
\end{lstlisting}%$

\noindent
which looks for a line of this general form and also captures the first number
before an \verb|f|. This first number is used as the number of columns of
displacements in the next section and signals the beginning of that section.

\subsection{Displacements}

This section is composed of displacements in the first $N-1$ columns, followed
by an optional column of energies. If the number of columns minus one is equal
to the number parsed in the aforementioned regular expression, the first $N-1$
columns are taken as displacements, while the last column is taken as the
corresponding energy. Otherwise, all $N$ columns are taken as part of the
displacement matrix. This allows for only the displacements to be input as part
of a template.

\begin{lstlisting}
 -0.00500000 -0.00500000 -0.01000000      0.000128387078
 -0.00500000 -0.00500000  0.00000000      0.000027809414
 -0.00500000 -0.00500000  0.01000000      0.000128387078
 -0.00500000 -0.01000000  0.00000000      0.000035977201
 -0.00500000 -0.01500000  0.00000000      0.000048243883
 -0.00500000  0.00000000 -0.01000000      0.000124321064
 -0.00500000  0.00000000  0.00000000      0.000023720402
 -0.00500000  0.00000000  0.01000000      0.000124321065
 -0.00500000  0.00500000 -0.01000000      0.000124313373
\end{lstlisting}

\subsection{Unknowns}

This section simply lists the number of columns in the matrix that follows. This
is necessary since the exponents are wrapped after 16 columns, so it would be
possible for all of them to blend into a single row. This section looks like:

\begin{lstlisting}
UNKNOWNS
  22
\end{lstlisting}

\subsection{Exponents}

This section gives the exponents for the polynomial equation to be fit. It has
the form given in the listing below.

\begin{lstlisting}
FUNCTION
   0    1    0    2    1    0    0    3    2    1    0    1    0    4    3    2
   1    0    2    1    0    0
   0    0    1    0    1    2    0    0    1    2    3    0    1    0    1    2
   3    4    0    1    2    0
   0    0    0    0    0    0    2    0    0    0    0    2    2    0    0    0
   0    0    2    2    2    4
\end{lstlisting}

\subsection{Stationary point}

This section requests a refitting of the energies to a known stationary point.
It looks like:

\begin{lstlisting}
STATIONARY POINT
     -0.000045311426     -0.000027076533      0.000000000000     -0.000000002131
\end{lstlisting}

\noindent{ Like the Displacement section above, the first $N-1$ columns are
  displacements used to bias the displacements in that section, while the last
  column is an energy used to bias each of the energies before performing the
  refitting. If this section is not present, a stationary point search is
  undertaken and a stationary point will be printed in the output. }

\section{Fitting}

The first task of ANPASS is to find the vector $F$ (of force constants in the
QFF problem) that minimizes $|XF - V|$. This minimization process is often
referred to as ``fitting'' since the discrete data points in the input are being
fit to the continuous function described by the $X$ matrix. With the matrix
formulation further described in Section~\ref{sec:over}, the fitting process
simply requires solving the
\href{https://en.wikipedia.org/wiki/Ordinary_least_squares}{ordinary least
  squares} equation given in Eqn.~\ref{eq:ols}.

\begin{equation}
  \label{eq:ols}
  F = (X^\intercal X)^{-1}X^\intercal V
\end{equation}

Since $X^\intercal X$ is always positive semidefinite, the Cholesky
decomposition can be used to compute its inverse. The rest of the solution is
then straightforward.

\section{Stationary Point}

\appendix

\section{Linear regression example}

The complete input for the linear regression example described in the Overview
is shown below. The Fortran version fails to run on this example, and the Go
version panics because it can't find a stationary point (since lines don't have
stationary points). However, the Go version will still write \verb|fort.9903|,
which contains the coefficients of the line.

\begin{lstlisting}[caption={anpass.in}]
!INPUT
TITLE
 H2O 2A1 F12-TZ
PRINT
   99
INDEPENDENT VARIABLES
   3
DATA POINTS
  11   -2
(1F12.8,f20.12)
  0.00000000      2.300000000000
  1.00000000      3.400000000000
  2.00000000      7.600000000000
  3.00000000      8.100000000000
  4.00000000      9.400000000000
  5.00000000     13.600000000000
  6.00000000     14.500000000000
  7.00000000     15.900000000000
  8.00000000     18.600000000000
  9.00000000     21.700000000000
 10.00000000     21.800000000000
UNKNOWNS
   2
FUNCTION
   1    0
END OF DATA
!FIT
!STATIONARY POINT
!END
\end{lstlisting}

\begin{lstlisting}[caption={fort.9903}]
    1    0    0    0      8.894019852120
    0    0    0    0      9.789763384464
\end{lstlisting}

\noindent
Again, the results found here have been converted to the units needed for a QFF,
so dividing them both by 4.359813653 gives the desired equation
$y = 2.04x + 2.24$.

\end{document}
