\documentclass{article}
\usepackage[margin=1in]{geometry}

\usepackage{listings}
\lstset{basicstyle=\ttfamily\small,breaklines=true}

\begin{document}

ANPASS is a program for performing least-squares analysis. It was originally
written in Fortran but has since been translated to Go and then to Rust by Brent
R. Westbrook. This documentation corresponds to the Rust version only.

\section{Input format}

The layout of the ANPASS input file is as follows:

\subsection{Header}

\begin{lstlisting}
!INPUT
TITLE
 H2O 2A1 F12-TZ
PRINT
   99
INDEPENDENT VARIABLES
   3
DATA POINTS
  69   -2
(3F12.8,f20.12)
\end{lstlisting}

\noindent
The only line read by the Rust version is the last. This line is parsed by the
regular expression:

\begin{lstlisting}
"(?i)^\s*\((\d+)f[0-9.]+,f[0-9.]+\)\s*$
\end{lstlisting}%$

\noindent
which looks for a line of this general form and also captures the first number
before an \verb|f|. This first number is used as the number of columns of
displacements in the next section and signals the beginning of that section.

\subsection{Displacements}

This section is composed of displacements in the first $N-1$ columns, followed
by an optional column of energies. If the number of columns minus one is equal
to the number parsed in the aforementioned regular expression, the first $N-1$
columns are taken as displacements, while the last column is taken as the
corresponding energy. Otherwise, all $N$ columns are taken as part of the
displacement matrix. This allows for only the displacements to be input as part
of a template.

\begin{lstlisting}
 -0.00500000 -0.00500000 -0.01000000      0.000128387078
 -0.00500000 -0.00500000  0.00000000      0.000027809414
 -0.00500000 -0.00500000  0.01000000      0.000128387078
 -0.00500000 -0.01000000  0.00000000      0.000035977201
 -0.00500000 -0.01500000  0.00000000      0.000048243883
 -0.00500000  0.00000000 -0.01000000      0.000124321064
 -0.00500000  0.00000000  0.00000000      0.000023720402
 -0.00500000  0.00000000  0.01000000      0.000124321065
 -0.00500000  0.00500000 -0.01000000      0.000124313373
\end{lstlisting}

\subsection{Unknowns}

This section simply lists the number of columns in the matrix that follows. This
is necessary since the exponents are wrapped after 16 columns, so it would be
possible for all of them to blend into a single row. This section looks like:

\begin{lstlisting}
UNKNOWNS
  22
\end{lstlisting}

\subsection{Exponents}

This section gives the exponents for the polynomial equation to be fit. It has
the form given in the listing below.

\begin{lstlisting}
FUNCTION
   0    1    0    2    1    0    0    3    2    1    0    1    0    4    3    2
   1    0    2    1    0    0
   0    0    1    0    1    2    0    0    1    2    3    0    1    0    1    2
   3    4    0    1    2    0
   0    0    0    0    0    0    2    0    0    0    0    2    2    0    0    0
   0    0    2    2    2    4
\end{lstlisting}

\subsection{Stationary point}

This section requests a refitting of the energies to a known stationary point.
It looks like:

\begin{lstlisting}
STATIONARY POINT
     -0.000045311426     -0.000027076533      0.000000000000     -0.000000002131
\end{lstlisting}

\noindent{ Like the Displacement section above, the first $N-1$ columns are
  displacements used to bias the displacements in that section, while the last
  column is an energy used to bias each of the energies before performing the
  refitting. If this section is not present, a stationary point search is
  undertaken and a stationary point will be printed in the output. }

\section{Function Fitting}



\end{document}